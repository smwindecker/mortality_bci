\documentclass[a4paper,11pt]{article}\usepackage[]{graphicx}\usepackage[]{color}
%% maxwidth is the original width if it is less than linewidth
%% otherwise use linewidth (to make sure the graphics do not exceed the margin)
\makeatletter
\def\maxwidth{ %
  \ifdim\Gin@nat@width>\linewidth
    \linewidth
  \else
    \Gin@nat@width
  \fi
}
\makeatother

\definecolor{fgcolor}{rgb}{0.345, 0.345, 0.345}
\newcommand{\hlnum}[1]{\textcolor[rgb]{0.686,0.059,0.569}{#1}}%
\newcommand{\hlstr}[1]{\textcolor[rgb]{0.192,0.494,0.8}{#1}}%
\newcommand{\hlcom}[1]{\textcolor[rgb]{0.678,0.584,0.686}{\textit{#1}}}%
\newcommand{\hlopt}[1]{\textcolor[rgb]{0,0,0}{#1}}%
\newcommand{\hlstd}[1]{\textcolor[rgb]{0.345,0.345,0.345}{#1}}%
\newcommand{\hlkwa}[1]{\textcolor[rgb]{0.161,0.373,0.58}{\textbf{#1}}}%
\newcommand{\hlkwb}[1]{\textcolor[rgb]{0.69,0.353,0.396}{#1}}%
\newcommand{\hlkwc}[1]{\textcolor[rgb]{0.333,0.667,0.333}{#1}}%
\newcommand{\hlkwd}[1]{\textcolor[rgb]{0.737,0.353,0.396}{\textbf{#1}}}%

\usepackage{framed}
\makeatletter
\newenvironment{kframe}{%
 \def\at@end@of@kframe{}%
 \ifinner\ifhmode%
  \def\at@end@of@kframe{\end{minipage}}%
  \begin{minipage}{\columnwidth}%
 \fi\fi%
 \def\FrameCommand##1{\hskip\@totalleftmargin \hskip-\fboxsep
 \colorbox{shadecolor}{##1}\hskip-\fboxsep
     % There is no \\@totalrightmargin, so:
     \hskip-\linewidth \hskip-\@totalleftmargin \hskip\columnwidth}%
 \MakeFramed {\advance\hsize-\width
   \@totalleftmargin\z@ \linewidth\hsize
   \@setminipage}}%
 {\par\unskip\endMakeFramed%
 \at@end@of@kframe}
\makeatother

\definecolor{shadecolor}{rgb}{.97, .97, .97}
\definecolor{messagecolor}{rgb}{0, 0, 0}
\definecolor{warningcolor}{rgb}{1, 0, 1}
\definecolor{errorcolor}{rgb}{1, 0, 0}
\newenvironment{knitrout}{}{} % an empty environment to be redefined in TeX

\usepackage{alltt}
\usepackage[osf]{mathpazo}
\usepackage{ms}
\usepackage[]{natbib}
\usepackage{gensymb}
%\raggedright

\newcommand{\stan}{\texttt{stan}}


\title{Modelling plant mortality: integrating intra and interspecific approaches}
\author{James S. Camac\textasteriskcentered$^1$, Richard G. Fitzjohn$^1$, Mark Westoby$^1$, Daniel S. Falster$^1$}
\affiliation{
$^1$ Department of Biological Sciences, Macquarie University, Sydney, NSW 2109, Australia\\
\textasteriskcentered Email for correspondence: \texttt{james.camac@gmail.com}\\
Word count: }
\date{}

\bibliographystyle{mee.bst}

\usepackage[title, titletoc, toc]{appendix}

\mstype{Research Article}
\runninghead{Incorporating traits into individual based mortality models}
\keywords{traits, wood density, growth rates, bayesian, survival analysis}



\IfFileExists{upquote.sty}{\usepackage{upquote}}{}
\begin{document}
\mstitlepage
\noindent
\parindent=1.5em
\addtolength{\parskip}{.3em}
\doublespacing
\linenumbers

\section{Summary}\label{abstract}

\section{Introduction}
Plant mortality is often modelled as a function of recent growth because it is thought to reflect the summed effects of many biotic and abiotic factors on a plants overall carbon budget \citep{Kobe:1995tw, Hawkes:2000ib, Keane:2001db, Wyckoff:2002ul}. As such, growth rates are considered a highly integrated individual-based predictor of plant mortality \citep{Hawkes:2000ib}. Within a species, the assumption of the growth-mortality relationship is that fast growing individuals (i.e. those that have larger carbon budgets) are more able to buffer stress and thus have higher survivorship relative to slow growing (i.e. low carbon budget) individuals \citep{Hawkes:2000ib, Keane:2001db}. Empirical studies examining this growth-mortality relationship have broadly supported this assumption, with many highlighting non-linear declines in mortality with increased growth \citep[e.g.][]{Kobe:1997vy, Wyckoff:2002ul,Vieilledent:2010fv}. Consequently, most dynamic vegetation models, including tree gap models \citep{Keane:2001db}, individual-based forest simulators \citep[e.g. SORTIE][]{Pacala:1996vm} and Dynamic Global Vegetation Models \citep[DGVMs; ][]{Woodward:2004ft} use some measure of growth coupled with a stochastic constant (i.e. baseline hazard or minimum chance of death) as the fundamental determinants of plant mortality \citep{Hawkes:2000ib, McDowell:2011dr}. 

However, due to a paucity of data on plant mortality, most dynamic vegetation models have implemented mortality algorithms using overly simplistic assumptions \citep{Hawkes:2000ib, Wyckoff:2000un, McDowell:2011dr}. One of the strongest assumptions these models make is that both baseline hazard and the growth-mortality relationship are constant across species \citep{Pacala:1996vm, Hawkes:2000ib, Wyckoff:2000un}. That is, all species have the same random chance of death and exhibit the same tolerance to low growth. This is despite growing empirical evidence highlighting that mortality rates vary considerably between species, even within the same community \citep{Kobe:1997vy, Wyckoff:2002ul, Kunstler:2005bn, Vieilledent:2010fv}. More recently, however, attempts have been made to examine this inter-specific variation by correlating species intrinsic growth and mortality rates \citep[e.g.][]{Poorter:2008iu, Wright:2010fl}. In contrast to intraspecific studies, these studies have shown that species with intrinsically faster growth rates exhibit higher mortality rates relative to slower growing species \citep{Loehle:1996vw, Poorter:2008iu, Wright:2010fl}. As such, species appear to exhibit a trade off between greater growth versus greater survival. Ecologists have attempted to explain this interspecific variation by using functional traits - attributes of species that are thought to influence vital rates (i.e. survival, growth and reproduction) and ultimately fitness \citep{Westoby:2002tb, Ackerly:2003eb, Falster:2015un}. 

Wood density, the biomass invested per unit wood volume, is one functional trait that has been consistently shown to be negatively correlated with tropical tree mortality \citep{Chave:2009iy, Poorter:2008iu, Kraft:2010kq, Wright:2010fl}. The underlying mechanism of this relationship is thought to be due to denser wood conveying enhanced physical resistance to random events such as stem breakage \citep{Putz:1983wu, vanGelder:2006exa, Chave:2009iy}, embolism \citep{Hacke:2001kj, Jacobsen:2005fx} and fungal and pathogen attack \citep{Augspurger:1984wx}. As such, it is hypothesized that species with higher wood density are likely to exhibit lower baseline hazards. However, research has also shown that wood density may increase tolerance to low growth rates, and thus alter the growth-mortality relationship. For example, species with high wood density are more tolerant shade \citep{vanGelder:2006exa} and competition \citep{Kunstler:2016km}. Studies have also shown that wood density may reduce drought induced carbon starvation \citep{McDowell:2011dr} and the susceptibility of ‘stressed’ individuals to opportunistic invertebrate, pathogen or fungal attack \citep{Augspurger:1984wx, McDowell:2011dr}.

What both intraspecific and interspecific studies have highlighted is that both individual (e.g. growth rates) and species (e.g. function traits) characteristics are required in order to develop generalizable models of plant mortality. However, to date, there has been little crossover of ideas between intraspecific and interspecific studies of plant mortality. For example, we are aware of only one study that uses individual based plant mortality empirical models to examine trait effects on baseline hazards \citep{AubryKientz:2013dg}, and none that examine how traits mediate the growth-mortality relationship.

In this study, we reconcile both intraspecific and interspecific theory by developing an individual based plant mortality model that incorporates individual based measurements of growth as well as the species trait wood density. Furthermore, we extend existing individual-based approaches \citep[e.g.][]{Kobe:1995tw} by simultaneously estimating both baseline hazards and the growth-mortality relationship (hereon in referred to as growth-dependent hazard). We parameterise this model using 15-years of individual dbh growth and mortality data from 203 tropical rainforest species encompassing 180500 individuals. Using this model,coupled with cross validation, we ask several questions: 1) Does simultaneously estimating baseline and growth-dependent hazards improve predictive capacity relative to the null model or growth-dependent only model 2) Does allowing both hazards to vary by species improve predictive capacity? 3) Does wood density predominately affect baseline hazards, or does it affect growth-dependent hazards or both? 4) What better predicts plant mortality, absolute dbh growth or basal area growth?

\clearpage

\section{Methods}

\subsection{Data}
We parameterise plant mortality models using individual growth and survival data collected from a 50-ha tropical rainforest plot on Barro Colorado Island (BCI), Panama (9.15\degree;N, 79.85\degree;W). The plot is predominately undisturbed old-growth rainforest. The climate is warm throughout the year with a mean annual temperature of 26.1\degree;. Rainfall is seasonal with most of the 2500 mm falling between April and November. The island is managed exclusively for field research by the Smithsonian Tropical Research Institute (STRI) and the permanent plot was established in 1980. Detailed descriptions on the flora, fauna, geology and climate of BCI can be found in (REFS).

Within the 50-ha BCI plot the diameter at breast height and survival status of all free-standing woody plants at least 1.3 m tall with a diameter greater or equal to 1 cm were recorded in 1981-1983, 1985, and every 5 years thereafter \citep{Condit:2012nz}. For the purpose of modelling mortality as a function of growth, we discarded data collected in 1982 and 1985 because diameter measurements were rounded to the nearest 5 mm whereas later censuses were rounded to the nearest millimetre. We also excluded individuals based on several rules: 1) those that did not survive at least two censuses (two being required to estimate growth rates); 2) those that were not consistently measured at 1.3 m above ground; 3) those that were multi-stemmed; or 4) those that resprouted or '\textit{returned from the dead}'. We also discarded species that do not exhibit secondary growth (e.g. palms and ferns), contained fewer than 10 individuals or did not contain an estimate of wood density. Extreme outliers: stems which grew more than 5 cm/yr or shrunk more than 25\% of their initial dbh were also removed. Using this dataset, we then randomly selected one census observation per individual. This was done in order to reduce computational requirements. In total, 180500 individuals of 203 species were included in our analysis.

Wood density for each species was estimated by coring trees located within 15 km of the BCI plot because increment borers are prohibited within the plot \citep{Wright:2010fl}. Cores were broken into pieces, each 5 cm long and specific gravity of each piece was determined by oven drying (100\degree C) and dividing by the fresh volume (as measured by water displacement). Wood density was then calculated as an area-weighted average (g cm$^-3$), where area refers to the annulus represented by each piece, assuming a circular trunk.

\subsection{Estimating true growth}
Field measurements almost always include observation error that can potentially manifest into biologically unrealistic scenarios of negative growth and ultimately biased conclusions \citep{Kery:2010tp}. In our dataset 0\% of observed growth rates were negative. We estimate this measurement error using another BCI dataset consisting of  randomly selected trees remeasured within 30 days (see XXXX). By assuming that no growth occurred between the two observations, we estimated the standard deviation of measurement error 
$\sigma_{error}$ was estimated to be 0.75, meaning that 95\% of the measurement errors were within +/- 1.47 cm of the true diameter. We then used this standard deviation of 0.75 in another Bayesian model to estimate the initial and final true dbh (i.e. observation - measurement error) for each individual. This was done by first modelling the observed initial ($dbh_{t-1}$) and final dbh ($dbh_{t}$) of each individual, $i$, as random realisations from a normal distribution centered on their respective true estimates (i.e. $\widehat{dbh1_i}$ and $\widehat{dbh2_i}$), with a standard deviation of 0.75 (i.e. $\sigma_{error}$).

\bibliography{refs}

\end{document}
